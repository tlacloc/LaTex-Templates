\documentclass[letterpaper, 12pt]{article}

%%%%%%% PACKAGES %%%%%%%%
\usepackage[utf8]{inputenc}
\usepackage[margin=2cm]{geometry}
\usepackage[spanish, es-nodecimaldot]{babel}
\usepackage{blindtext}
\usepackage{setspace}
\usepackage{graphicx}
\usepackage{amsfonts}
\usepackage{multirow}
\usepackage{mathrsfs}


% EASY MATH
\usepackage[thinc]{esdiff} %derivadas faciles
\usepackage{physics} %algunos simbolos de derivadas
\usepackage{amsmath}
\usepackage{amsthm}
\usepackage{amsbsy}
\usepackage{amssymb}
\usepackage{mathtools}
\usepackage{siunitx}

% DOCUMENT PROPERTIES
\usepackage{color}
\usepackage[hidelinks]{hyperref}
\usepackage{cite}
\usepackage{enumitem}
\usepackage{epsfig}
\usepackage{caption}
\usepackage{subcaption}
\usepackage[strict]{changepage}
\usepackage{booktabs}
\usepackage{float} 
%\usepackage[document]{ragged2e} %para justificado de párrafo
\usepackage{rotating}
\usepackage{tabularx}
\usepackage{ifthen}
\usepackage{commath}
\usepackage{esint}
\usepackage{mathtools}
\usepackage{listing}
\usepackage{multirow}
\usepackage{tabto}
\usepackage{csquotes}
\usepackage{xcolor}
\usepackage{advdate}
\usepackage[acronym, toc]{glossaries}
%\usepackage[siunitx]{circuitikzgit}
%\usepackage{xfrac}
\selectlanguage{spanish}

%path donde se encuentran las imagenes
\graphicspath{ {./figuras/} }

%-------------------------------------------------
%PORTADA
%-------------------------------------------------

\begin{document}
\begin{center}
\begin{Large}
{\color{white} mmmmmmm} \bigskip\\
\begin{figure}[H]	
\centering
\includegraphics[scale=0.9]{bannerfiuady.png}
\end{figure}\bigskip
\textsc{FACULTAD DE INGENIERÍA}\bigskip\\
Periodo agosto - diciembre 2021 \bigskip\\
{\color{white} mmmmmmm} \\
{\color{white} mmmmmmm} \bigskip\\
\textsc{Materia} \bigskip\\
Profesor \\
{\color{white} mmmmmmm} \bigskip\\
\textsl{Tarea } \bigskip\\
{\color{white} mmmmmmm} \bigskip\\
Erick Al. Casanova Cortés\\

\normalsize
Ingeniería Física\\
\small
a15014866@alumnos.uady.mx\\
\Large
\rule{0.5\paperwidth}{0.5pt} \\ \bigskip
{\AdvanceDate\today}
\end{Large}
\end{center}
\newpage


\pagenumbering{arabic}
%\clearpage\maketitle
\setcounter{page}{1}


%%%%%%%%%% INDEX
\tableofcontents
\listoffigures
\listoftables

\newpage
\pagenumbering{arabic}

%%% Contenido principal %%%%

%-------------------------------------------------
%Inicio del documento
%-------------------------------------------------



\Large
\textit{ \textbf{Resumen}}
\normalsize

\section{Objetivo}
En esta sección se describe el objetivo general de la práctica, así como los objetivos específicos que se pretende lograr con la implementación del dispositivo o circuito.\\

\textit{ \textbf{Objetivo general}}

Diseñar un modelo didáctico capaz de generar campos magnéticos constantes y variables en el tiempo, controlado por el usuario a través de una interfaz gráfica amigable e intuitiva.
En el párrafo de arriba hay un ejemplo de un objetivo redactado. Los puntos más importantes son: 

\begin{itemize}

    \item debe contener el propósito final del circuito o dispositivo (responde a la pregunta ¿para qué sirve?),
    \item debe ser una sola oración,
    \item debe tener solamente un verbo.
    \end{itemize}

\textit{ \textbf{Objetivos específicos}}
\begin{itemize}

    \item[-] Algún objetivo secundario que sirve para poder lograr el objetivo general,
    \item[-] Algún objetivo secundario que sirve para poder lograr el objetivo general,
    \item[-] Algún objetivo secundario que sirve para poder lograr el objetivo general.
    \end{itemize}

\section{Itroducción}
Contenido de la sección INTRODUCCIÓN; al igual que el resto del documento, debe respetar el formato de fuente, márgenes e interlineado. En esta sección se presenta qué cosas (instrumentos, métodos, etc.) existen para resolver problemáticas similares a la presentada en el objetivo.

\section{Sección X}
Todas las figuras deben estar en buena calidad, ser perfectamente visibles. Todas las imágenes deben llevar pie de foto con número y nombre; debe hacerse referencia a la figura dentro del texto.

\section{Sección Y}
Las tablas que se incluyan en el reporte deberán estar numerados y nombradas en la parte superior de la tabla.  Deberá hacerse referencia a la tabla dentro del cuerpo del texto.

\section{Conclusiones}
En esta sección se presentan las conclusiones derivadas del proyecto final.

%%%%%%%%%%%%%%%%%%%%%%%%%%%%%%%%%%%%%%%%%%%%%%%%%%%%%%%%%%%%%%
\section{Referencias}
\bibliographystyle{ieeetr}
\bibliography{referencias}

%-------------------------------------------------
%Final del documento
%-------------------------------------------------

\end{document}

